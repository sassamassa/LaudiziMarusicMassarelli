\documentclass{article}
\usepackage{graphicx} % Required for inserting images
\graphicspath{{images/}} %configuring the graphicx package
\usepackage{listings}
\usepackage{xcolor}
\usepackage[utf8]{inputenc}
\usepackage{babel}
\usepackage[margin=2.5cm]{geometry}
\usepackage{todonotes} %for to do notes
\usepackage{enumitem} %for customizable enumerations


%definition of variables for uniformity of naming
\newcommand{\admin}{admin }
\newcommand{\admins}{admins }
\newcommand{\adminGroup}{admin goup }
\newcommand{\adminGroups}{admin groups }
\newcommand{\participant}{participant }
\newcommand{\participants}{participants }
\newcommand{\group}{groups }
\newcommand{\groups}{groups }
\newcommand{\user}{user }
\newcommand{\users}{users }


%notes: respect the correct formatting of the Latex code in order to increase readability.
% each section should start at the leftmost area of the screen and each subsection should be indented.
%use to do blocks to spot areas that needs further development or examination
%always refer to the sytem as system, not website or other specific names because this will result in a premature desing choice




\title{RASD\\ CodeKataBattles}
\author{Federica Laudizi, Antonio Marusic, Sara Massarelli}
\date{November 2023}

\begin{document}


\maketitle

\tableofcontents

\todo{look for a way to identify spelling mistakes}

\begin{abstract}
    This Requirements Analysis and Specification Document (RASD) is going to takle objectives, requirements (functional and non functional) for the project CodeKataBattles.
    
    It will be presented the scope, the functionalities and the domain of such project and we will give a comprehensive overview of the interaction with users and external components and the performance expected from the system.

    This document is going to be a reliable point of reference for developers since it will define clearly the functionalities of the system and it will give precise guidelines during the validation and verification process.

    The profound goal of this document is to give a precise overview of the product to be in order to give the stakeholders a general understanding of usage scenarios and avoiding changes of directions during the design and implementation part of the system development.    
\end{abstract}

\section{Introduction} 
    \subsection{Purpose}
        This system allows students to boost their learning by participating in
        code challenges called kata battles organised by teachers and independent tutors.\\

        Such challenges will be part of tournaments so that it is possible for students to compete in multiple katas and seeing their position in a leaderbord of all participants.
        The system allows for the creator of the tournament to define a set of rules regarding the number of participants and the size of the groups of students and it gives the possibility to delegate the creation of the katas and the managing of the tournament also to other teachers.\\

        The system offers a platform integrated with GitHub where teachers can publish the kata and define the test cases that the code provided by the students must pass and assess student's submissions.
        The students on the other hand, can submit their code and see immediately
        its performances. The system assigns a score to each submission
        based on functional and non functional aspects of the code so that
        students can compare their work with other participants.\\
        
        Moreover, the system promotes global participation in kata battles, since it is possible to join battles organised by educators from all around the world.
        It is worth noticing that, not only certified teachers can create tournaments, but also independent tutors, allowing them to engage their students in productive competitions. 
        
            \subsubsection{Goals}
                The system must satisfy this goals:

                Actors definitions:
                \begin {itemize}
                    \item Admin: admins are the teachers, educators or tutors that created the tournament or were accepted by the creator as collaborators. Admin users have the previlege for that tournament to manage the tournament settings and parameters and to publish new kata battles
                    \item Participants: are the students that enroll in a tournament. 
                    \todo{better explain the subdivision in groups, maybe we don't have only the concept of participant but we have the concept of group and participants are the components of the group}
                \end {itemize}



                
                \begin{enumerate}[label=\textbf{G\arabic*}:]
                    \item Participants can participate in a tournament.
                    \item Participants can submit their code through GitHub.
                    \item Participants can view the results of their submission as a score from 0 to 100.
                    \item Participants can see the scores of other participants and their position in the leaderboard, whether they are in the same tournament or not.
                    \item Participants can see all tournaments currently running.
                    \item Admins can invite collaborators to the tournament that will be considered as admins once they accept the invite.
                    \item Users can request to become admins.
                    \item Admins can accept user requests to become admins.
                    \item Admins can accept user requests to participate in a tournament.
                    \item Admins and collaborators can create kata battles.
                    \item Admins can change the parameters for tournaments they created (e.g., make the tournament open, set the max number of participants, set the group size, set the duration of each kata battle).
                    \todo{Better define what the parameters are.}
                    \item Admins can...
                \end{enumerate}
                \todo {to be continued}

    \section{Overall caracteristics}
        \subsection{Product functions}
        \paragraph*{Visualization of the system}
            \subsubsection{Visualisation of the badges}
            \subsubsection{Vedere la rank}
            \subsubsection{Vedere i tournament}
            \subsubsection{Visualize partecipant}
        
        \paragraph*{Modification of the system}
            \subsubsection{Push of the code}
            \subsubsection{Signup and login same for everyone}
            \subsubsection{Signup and login same for everyone}
            \subsubsection{Create a tournament and Set parameters}
            \subsubsection{Invite}
                \paragraph*{colleague}

                \paragraph*{group member}
            \subsubsection{Accept invite}
                \paragraph*{colleague}

                \paragraph*{group member}
            \subsubsection{Decline invite}
            \subsubsection{Create repository}
            \subsubsection{Participate in a tournament}
            \subsubsection{Participate in a battle}
            \subsubsection{A°°°Send notification}
                The system has the possiblity of sending notification to \users. There are multiple kinds of notifications that serves the purpose of updating \users on the status of the tournanaments they are involved in and updating them on incoming invitations in tournaments, group or invitations as collaborators in the \adminGroup.
            
            
                \paragraph{Tournament created}
                    \todo{Since we espect plenty of tournament to be created daily maybe it is a good idea to give the possibility to disable or filter this kind of notifications. It is sensible also to give \admins the possibility of creating closed tournaments, accessible only through invitation. Are this closed tournaments visible in the list of tournaments?}

                    When a new tournament is created, all students subscribed to the CKB platform are notified.

                \paragraph{Battle created}
                    When someone in the \adminGroup of a tournament creates a kata battle, every \participant of that tournament is notified.
                \paragraph*{Send link}
                    When the registration deadline expires and the platform creates the GitHub repository the system sends a notification containing the link to all \participants of that tournament.
                \paragraph*{End of battle}
                    Once the consolidation stage finishes, all students participating in the battle are notified with the updated ranks.
                
                \paragraph{Tournament closure}
                    \todo{when does a tournament finishes?}
                    when a tournament finishes all the \participants are notified

                \paragraph*{Invite}
                    \todo{this section isn't specifically required but it makes sense considering the fact that one can invite team members so you have to someway understand that you have been invited}
                    The invitations can be received in different cases
                    \begin {itemize}
                        \item an \user can be invited in a \group participating a tournament
                        \item an \user can be invited in the \adminGroup of a tournament
                        \item an user can be invited in a tournament
                            \todo {is the group invited in the tournament of the single \user?}
                    \end {itemize}
                    
            \subsubsection{A°°°Run tests}
                \todo {probably this section doesn't belong here}
                Each git push performed on the challenge reposistory by a \group participating in the kata battle triggers the platform that builds the project and runs the tests.
                \todo{where and how are the test runned?}

            \subsubsection{A°°°Assign score automatically}
                When the test of a submission has runned a score is assigned by the system

                \todo{where should this section go?}
                Scoring method (copy and paste of the instructions)
                The score is a natural number between 0 and 100 determined by considering some mandatory factors evaluated in a fully automated way, and optional factors evaluated manually by educators. Mandatory automated evaluation includes:
                • functional aspects, measured in terms of number of test cases that pass out of all test cases (the
                higher the better);
                • timeliness, measured in terms of time passed between the registration deadline and the last
                commit (the lower the better);
                • quality level of the sources, extracted through static analysis tools that consider multiple aspects
                such as security, reliability, and maintainability (the higher the better). Aspects are selected by the
                educator at battle creation time. Optional manual evaluation includes:
                • personal score assigned by the educator, who checks and evaluates the work done by students (the
                higher the better).
                The CKB platform automatically updates the battle score of a team as soon as new push actions on GitHub are performed
            \subsubsection{A°°°Consolidation stage: manually evaluate repo}
                When the submission deadline expires, the kata battle enters in a consolidation state where the \admins can review the repository and perform further evaluations. The scores can be modified manually. When this stage finishes notifications are sent.
            \subsubsection{A°°°Create a Kata Battle}
                The system allows \admins to create a Kata Battle
                \todo {define a kata battle}. The process involves asking a
                textual description of the problem to solve by the \groups and a software project with build automation scripts whose structure depends on the programming language of choiche for the kata. This project must contain a set of test cases, otherwise it won't be possible to perform the evaluation of the submissions. 
                \todo {can katas allow for different programming languages?}.
                After that the sytem prompts for some configurations parameters for the kata, such as minimum and maximum number of students per group, a registration deadline, a final submission deadline
                and additional configurations for scoring
                \todo {pay attention to the fact that scoring criterions are different based on the kata, so are not a property of the tournament}\\

                When the system receives in input the data required it performes some validity checks
                \todo{what are the necessary validity checks? what can't be validated and so has to be put in the constraints?}.
                 Later, it will ask for confirmation and the kata will be created.
                \todo{ Notifications will be sent?}
                
            \subsubsection{A°°°Create Badge}
                The system allow \admins of a tournament to set badges. In order to create a new badge the admin has to write a formula that involves variables provided by the system. Then he/she/they/otherPronouns can specify for how long is the badge obtainable. If the badges is of type boolean it has to specify how many badges are available. In this phase some validity checks are performed, for example if the formula is always false is rejected, or if the time is before now or if it is after the end of the tournament is rejected
                \todo{is there an end time for the tournament?}
                
                \todo {explain what variables are} 
                \todo{explain what badges are}

                \begin{itemize}
                    \item Top kind badges (only one badges, the formula returns the best \participant /group)
                    \item Boolean kind badges (are given if a condition is satisfied)
                \end {itemize}

            
                • Title: tournament participant
                %o Rules: { tot_attended_battles > 0 }
                • Title: top committer
                %o Rules: { tot_commits_student == max_tot_commits }
                %Where tot_attended_battles,tot_commits_studentandmax_tot_commitsare pre-defined variables that represent, respectively, the total number of battles the student has been involved in, the number of commits carried out by the student and the maximum total number of commits considering all students.

                \todo {educators can create new badges and define new rules as well as new variables associated with them (it is written in the instructions)}

                Badges can be visualized by all users. In particular, both students and educators can see collected badges when they visualize the profile of a student.
            
            \subsubsection{Evaluate Badges}
                \todo {probably this section doesn't belong here}
                \todo{how often are the badges checked? every commit? can the system identify when is sensible to check a badge based on the kind of badge?}


    \subsection{Assumptions, dependencies and constraints}
        The \groups correctly create an automated workflow on GitHub.
        \todo {how does the system reacts if this doesn't happens? can the system detect such mistakes?}


\section{Effort spent}
    Antonio Marusic\\
    section 1: 2h\\
    section 2: 2:30h\\
    
\end{document}